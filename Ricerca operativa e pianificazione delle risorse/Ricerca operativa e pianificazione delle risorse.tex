\documentclass[12pt]{article}

\usepackage{booktabs}% http://ctan.org/pkg/booktabs
\usepackage[utf8]{inputenc}
\usepackage{changepage}
\usepackage{pgfplots}
\usepackage{amssymb}
\usepackage{xcolor}
\usepackage{hyperref}
\usepackage{listings}
\usepackage[T1]{fontenc}
\usepackage[utf8]{inputenc}
\usepackage{adjustbox}
\usepackage{amsmath}
\usepackage{mathtools}
\usepackage{biblatex}
\lstset{
  language=Python,
  numbers=left,
  numberstyle=\tiny,
  stepnumber=1,
  numbersep=5pt,
  tabsize=4,
  basicstyle=\ttfamily,
  columns=fullflexible,
  keepspaces,
}
\hypersetup{
    colorlinks,
    citecolor=black,
    filecolor=black,
    linkcolor=black,
    urlcolor=black
}

% Set page size and margins
% Replace `letterpaper' with `a4paper' for UK/EU standard size
\usepackage[letterpaper,top=2cm,bottom=2cm,left=3cm,right=3cm,marginparwidth=1.75cm]{geometry}

% Useful packages
\usepackage{amsmath}
\usepackage{mathtools}
\usepackage{graphicx}
\newenvironment{para}{\begin{adjustwidth}{13mm}{}}{\end{adjustwidth}}

\newcommand\tab[1][1cm]{\hspace*{#1}}

\newcommand{\tabitem}{\llap{\textbullet}}
\newcommand{\Hsquare}{%
\text{\fboxsep=-.2pt\fbox{\rule{0pt}{1ex}\rule{1ex}{0pt}}}%
}

\newtheorem{Definizione}{Definizione}[subsection]
\newtheorem{Lemma}{Lemma}[subsection]
\newtheorem{Teorema/Definizione}{Teorema/Definizione}[subsection]
\newtheorem{Corollario}{Corollario}[subsection]
\newtheorem{Teorema}{Teorema}[subsection]
\newtheorem{Proposizione}{Proposizione}[subsection]
\newtheorem{Notazione}{Notazione}[subsection]
\newtheorem{Commento}{Commento}[subsection]
\newtheorem{Dimostrazione}{Dimostrazione}[subsection]
\newtheorem{Osservazione}{Osservazione}[subsection]
\newtheorem{Nota}{Nota}[subsection]

\title{Ricerca operativa e pianificazione delle risorse}
\author{spitfire}
\date{A.A. 2023-2024}
\begin{document}
\begin{figure}
    \centering
    \includegraphics[width=0.35\textwidth]{Images/Logo scienze bicocca.png}
\end{figure}

\vspace{10cm}
\date{A.A. 2024-2025}


\maketitle

\newpage

\tableofcontents
\newpage

\section{Prerequisiti di Algebra Lineare}
\subsection{Matrici e vettori}
Una matrice è una tabella contenente numeri.
Se la tabella è costituita da $m$ righe e $n$ colonne si parla
di una matrice  $m \times n$. 
Una matrice viene detta \textbf{matrice quadrata} se il numero di righe
e colonne coincidono. \newline
Una matrice $1 \times m$ viene detto \textbf{vettore riga m-dimensionale} \newline
Una matrice $m \times 1$ viene detto \textbf{vettore colonna m-dimensionale}. \newline
La notazione maggiormente utilizzata per indicare una matrice è
$$A = [a_{ij}]$$
Con $a_{ij}$ elemento generico della i-esima riga e j-esima colonna della matrice $A$.
Se $A = [a_{ij}]$ è una matrice $m \times n$, la matrice $n \times m$
$$A^T=[a_{ij}]$$
viene detta \textbf{matrice trasposta} della matrice $A$.

Se $A = [a_{ik}]$ è una matrice $m \times p$ e $B = [b_{kj}]$ è una matrice $p \times n$ la loro
\textbf{matrice prodotto} è $m \times n$ e definita come:
$$A \cdot B = C = [c_{ij}] \; con \; c_{ij} = \sum_{k = 1}^{p} a_{ik} \cdot b_{kj}$$
Date due matrici $m \times n, A = [a_{ij}]$ e $B = [b_{ij}]$, la loro \textbf{matrice somma} è definita come segue:
$$A+B=C=[c_{ij}] \; con \; c_{ij} = a_{ij} + b_{ij}$$
La \textbf{moltiplicazione} di una \textbf{matrice A per una costante $\alpha$} fornisce come risultato quanto segue:
$$\alpha \cdot A = [\alpha \cdot a_{ij}]$$
Siano $v_1, v_2, ..., v_n$ n vettori, riga o colonna; essi vengono detti
\textbf{linearmente indipendenti} tra loro se, prendendo $n$ coefficienti $a_1, a_2, ..., a_n$ la seguente uguaglianza
$$a_1 \cdot v_1 + a_2 \cdot v_2 + ... + a_n \cdot v_n = 0$$
risulta verificata solo se $a_1 = a_2 = ... = a_n = 0$. \newline
Al contrario, se esistono coefficienti $a_1, a_2, ..., a_n$ non tutti nulli per cui
$$a_1 \cdot v_1 + a_2 \cdot v_2 + ... + a_n \cdot v_n = 0$$
i vettori $v_1, v_2, ..., v_n$ sono detti \textbf{linearmente dipendenti}. \newline
Un insieme di $n$ vettori ad $n$ dimensioni linearmente indipendenti costituisce una \textbf{base per uno spazio a n dimensioni}.
Se un insieme di vettori $v_1, v_2, ..., v_n$ costituisce una base per uno spazio ad $n$ dimensioni, allora ogni vettore $x$ che appartiene
a quello spazio è \textbf{combinazione lineare dei vettori della base}. \newline
Una matrice quadrata $m \times m$ si dice \textbf{matrice singolare} se l'insieme degli $m$ vettori riga (o colonna), ottenuti considerando
ogni riga (o colonna) come un vettore, è \textbf{linearmente dipendenti}.
Se, viceversa, l'insieme degli $m$ vettori è linearmente indipendente, la matrice si dice \textbf{matrice non singolare}. \newline
Una matrice quadrata $A = [a_{ij}]$ con $a_{ij} = 0$ per ogni $i \neq j$ viene detta \textbf{matrice diagonale}. \newline
La matrice diagonale $A = [a_{ij}]$, con $a_{ii} = 1$ per ogni $i$ viene detta \textbf{matrice identità}, solitamente indicata con $I$.
Se $A$ NON è una matrice singolare, allora esiste una matrice $A^{-1}$ detta \textbf{matrice inversa} della matrice $A$, tale per cui vale la
seguente relazione di uguaglianza:
$$A \cdot A^{-1} = A^{-1} \cdot A = I$$
Il \textbf{determinante} di una matrice quadrata $A$ si indica con $det(A)$ ed è un numero (esiste solo per matrici quadrate), nel caso
specifico di una matrice $2 \times 2$ si definisce come segue:
$$det(A) = det\begin{pmatrix}
    a_{11} & a_{12} \\
    a_{21} & a_{22}
\end{pmatrix} = a_{11} \cdot a_{22} - a_{12} \cdot a_{21}$$
Il determinante di una matrice quadrata $A$ $m \times m$ si ottiene utilizzando la seguente regola ricorsiva, detta \textbf{formula di Laplace}:
Se $A_{ij}$ è la matrice $(m-1) \times (m-1)$, ottenuta togliendo la i-esima riga e la j-esima colonna di A, il determinante di A risulta:
$$det(A) = \sum_{j=1}^{m} (-1)^{i+j} \cdot a_{ij} \cdot det(A_{ij}) \; (formula \; per \; righe)$$
$$det(A) = \sum_{i=1}^{m} (-1)^{i+j} \cdot a_{ij} \cdot det(A_{ij}) \; (formula \; per \; colonne)$$
Se la matrice è singolare, allora $det(A) = 0$. \newline
Una matrice quadrata $A$ ammette inversa se e solo se non è singolare.
\subsection{Equazioni lineari}
Un' \textbf{equazione lineare} nelle variabili $x_1, x_2, ..., x_n$ è un'equazione nella seguente forma:
$$a_1 \cdot x_1 + a_2 \cdot x_2 + ... + a_n \cdot x_n = b$$
dove $a_1, a_2, ..., a_n$ e $b$ sono delle costanti.
Si dice \textbf{soluzione dell'equazione} un qualsiasi vettore $|y_1, y_2, ..., y_n| \in \mathbb{R}^n$ tale che:
$$a_1 \cdot y_1 + a_2 \cdot y_2 + ... + a_n \cdot y_n = b$$
Un \textbf{sistema di m equazioni lineari in n variabili} è definito come segue:
$$\begin{cases}
    a_{11} \cdot x_1 + a_{12} \cdot x_2 + ... + a_{1n} \cdot x_n = b_1 \\
    a_{21} \cdot x_1 + a_{22} \cdot x_2 + ... + a_{2n} \cdot x_n = b_2 \\
    ... \\
    a_{m1} \cdot x_1 + a_{m2} \cdot x_2 + ... + a_{mn} \cdot x_n = b_m
\end{cases}$$
dove $a_{ij}$ e $b_{j}$, $i = 1,...,n$; $j = 1,...,m$ sono costanti.

\end{document}