\documentclass[12pt]{article}

\usepackage{booktabs}% http://ctan.org/pkg/booktabs
\usepackage[utf8]{inputenc}
\usepackage{changepage}
\usepackage{pgfplots}
\usepackage{amssymb}
\usepackage{xcolor}
\usepackage{hyperref}
\usepackage{listings}
\usepackage[T1]{fontenc}
\usepackage[utf8]{inputenc}
\usepackage{adjustbox}
\usepackage{amsmath}
\usepackage{mathtools}
\usepackage{biblatex}
\lstset{
  language=Python,
  numbers=left,
  numberstyle=\tiny,
  stepnumber=1,
  numbersep=5pt,
  tabsize=4,
  basicstyle=\ttfamily,
  columns=fullflexible,
  keepspaces,
}
\hypersetup{
    colorlinks,
    citecolor=black,
    filecolor=black,
    linkcolor=black,
    urlcolor=black
}

% Set page size and margins
% Replace `letterpaper' with `a4paper' for UK/EU standard size
\usepackage[letterpaper,top=2cm,bottom=2cm,left=3cm,right=3cm,marginparwidth=1.75cm]{geometry}

% Useful packages
\usepackage{amsmath}
\usepackage{mathtools}
\usepackage{graphicx}
\newenvironment{para}{\begin{adjustwidth}{13mm}{}}{\end{adjustwidth}}

\newcommand\tab[1][1cm]{\hspace*{#1}}

\newcommand{\tabitem}{\llap{\textbullet}}
\newcommand{\Hsquare}{%
\text{\fboxsep=-.2pt\fbox{\rule{0pt}{1ex}\rule{1ex}{0pt}}}%
}

\newtheorem{Definizione}{Definizione}[subsection]
\newtheorem{Lemma}{Lemma}[subsection]
\newtheorem{Teorema/Definizione}{Teorema/Definizione}[subsection]
\newtheorem{Corollario}{Corollario}[subsection]
\newtheorem{Teorema}{Teorema}[subsection]
\newtheorem{Proposizione}{Proposizione}[subsection]
\newtheorem{Notazione}{Notazione}[subsection]
\newtheorem{Commento}{Commento}[subsection]
\newtheorem{Dimostrazione}{Dimostrazione}[subsection]
\newtheorem{Osservazione}{Osservazione}[subsection]
\newtheorem{Nota}{Nota}[subsection]


\title{Analisi e Progettazione del Software}
\author{spitfire}
\date{A.A. 2023-2024}
\begin{document}
\begin{figure}
    \centering
    \includegraphics[width=0.35\textwidth]{Images/Logo scienze bicocca.png}
\end{figure}

\vspace{10cm}
\date{A.A. 2023-2024}


\maketitle

\newpage

\tableofcontents
\newpage

\section{Introduzione}
Che cos'è il \textbf{software}? Esso è \textbf{un programma per computer} unito alla \textbf{documentazione ad esso associata},
la quale specifica e comprende \textbf{requisiti, modelli di progetto, manuale utente,...} \newline
I prodotti software possono essere:
\begin{itemize}
    \item \textbf{Generici}: sviluppati per un ampio insieme di clienti (elaboratori di testo, database,...)
    \item \textbf{Personalizzati} (custom): sviluppati per un singolo cliente in base alla sue esigenze specifiche
\end{itemize}
Un nuovo prodotto software può essere \textbf{creato da zero, personalizzando software già esistenti o riusando parti o software già esistente}.
Le caratteristiche essenziali di un buon software sono:
\begin{center}
    \includegraphics[width = 0.80\textwidth]{Images/1.png}
\end{center}
\subsection{Introduzione all'ingegneria del software}
Che cos'è \textbf{l'ingegneria del software}? \textbf{L'ingegneria del software} è una disciplina ingegneristica che si occupa
di tutti gli aspetti della produzione del software di buona qualità, dalle \textbf{prime fasi della specifica del sistema fino alla manutenzione del sistema}
dopo la messa in uso. Vediamo cosa si intende per \textbf{disciplina ingegneristica} e "\textbf{Tutti gli aspetti della produzione del software}":
\begin{itemize}
    \item \textbf{Disciplina ingegneristica}: Utilizzare metodi e teorie \textbf{appropriati} per risolvere i problemi tenendo conto dei vincoli \textbf{organizzativi e finanziari}
    \item \textbf{Tutti gli aspetti della produzione del software}: Non solo il \textbf{processo tecnico di sviluppo}. Anche la \textbf{gestione del progetto} e lo sviluppo di \textbf{strumenti}
    ,metodi ecc... per supportare la produzione del software
\end{itemize}
\newpage
\noindent
La disciplina dell'ingegneria del software si occupa di:
\begin{center}
    \includegraphics[width = 0.80\textwidth]{Images/2.png}
\end{center}
\subsection{La crisi del software}
Il termine \textbf{crisi del software} (o software crisis) è usato nell'ambito dell'ingegneria del software per descrivere l'impatto della \textbf{rapida crescita} della potenza degli elaboratori
e la \textbf{complessità} dei problemi che dovevano esseri affrontati. Le parole chiavi della software crisis erano \textbf{complessità, attese e cambiamento}. Il concetto di software crisis emerse negli
anni '60.
\begin{center}
    \includegraphics[width = 0.80\textwidth]{Images/3.png}
\end{center}
\begin{center}
    \includegraphics[width = 0.80\textwidth]{Images/4.png}
\end{center}
Le cause della crisi del software erano legate alla \textbf{complessità dei processi software} e alla \textbf{relativa immaturità dell'ingegneria del software}
Per superare la crisi infatti si dovettero introdurre:
\begin{itemize}
    \item \textbf{Management}
    \item \textbf{Organizzazione}, attraverso \textbf{analisi e progettazione}
    \item \textbf{Teorie e tecniche} come la \textbf{programmazione strutturata e ad oggetti}
    \item \textbf{Strumenti}, come gli IDE
    \item \textbf{Metodologie}, tra cui il \textbf{modello a cascata e il modello agile}
\end{itemize}
\subsection{Analisi e progettazione}
Che cosa sono \textbf{analisi e progettazione}? \newline
\textbf{L'analisi} enfatizza un'\textbf{investigazione del problema e dei requisiti} invece che una soluzione: per esempio, se si vuole realizzare un nuovo sistema di trading online,
bisognerà capire \textbf{come questo sistema verrà utilizzato} e \textbf{quali sono le sue funzioni}. "Analisi" è un termine ampio con più accezioni, tra cui:
\begin{itemize}
    \item \textbf{Analisi dei requisiti}, cioè un'investigazione dei requisiti del sistema
    \item \textbf{Analisi orientata agli oggetti}, cioè un'investigazione degli oggetti di dominio
\end{itemize} 
La \textbf{progettazione} enfatizza una soluzione \textbf{concettuale} (software e hardware) che \textbf{soddisfa i requisiti}, anziché la relativa implementazione. Per esempio, la descrizione di uno schema di base di dati
e di oggetti software. Nella progettazione vengono spesso \textbf{esclusi dettagli di basso livello o "ovvi"} (o almeno "ovvi" per coloro a cui è destinato il software). \newline
Infine i progetti possono essere \textbf{implementati} e la loro implementazione (ovvero il codice) esprime il progetto realizzato vero e completo.
Come nel caso dell'analisi, anche "progettazione" è un termine con più accezioni, tra cui:
\begin{itemize}
    \item \textbf{Progettazione orientata agli oggetti}
    \item \textbf{Progettazione di basi di dati}
\end{itemize}
L'analisi e la progettazione possono essere riassunti con la seguente frase:
\begin{center}
    \textbf{Fare la cosa giusta}(analisi) \textbf{e fare la cosa bene}(progettazione)
\end{center}
\subsubsection{Analisi e progettazione orientata agli oggetti}
Durante \textbf{l'analisi orientata agli oggetti} c'è un enfasi sull'\textbf{identificazione} e la \textbf{descrizione degli oggetti}, o dei \textbf{concetti}, nel \textbf{dominio del problema}.
Per esempio, nel caso di un sistema informatico per voli aerei, alcuni dei concetti possono essere \textit{Aereo, Volo} e \textit{Pilota}. \newline
Durante \textbf{la progettazione orientata agli oggetti} (o più semplicemente \textbf{progettazione a oggetti}) l'enfasi è sulla \textbf{definizione di oggetti software} e sul \textbf{modo in cui questi collaborano
per soddisfare i requisiti}. Per esempio un oggetto software \textit{Aereo} può avere un attributo \textit{codiceDiRegistrazione} e un metodo \textit{getVoliEffettuati}. \newline
\begin{center}
    \includegraphics[width = 0.70\textwidth]{Images/5.png}
\end{center}
Infine durante \textbf{l'implementazione} o la \textbf{programmazione orientata agli oggetti}, gli oggetti progettati vengono implementati, per esempio implementando la classe \textit{Aereo} in un linguaggio ad oggetti.
Dunque, analisi e progettazione \textbf{hanno obbiettivi diversi che vengono perseguiti in maniera diversa}. Tuttavia, come mostrato dall'esempio sopra, esse sono \textbf{attività fortemente sinergiche} che sono \textbf{correlate fra loro}
e con le \textbf{altre attività dello sviluppo del software}.
\subsection{Introduzione ai diagrammi e ai passi fondamentali dello sviluppo software}
Vediamo una breve introduzione dei \textbf{vai diagrammi e dei passi fondamentali} legati allo sviluppo software.
\begin{center}
    \includegraphics[width = 0.80\textwidth]{Images/6.png}
\end{center}
\subsubsection{Definizione dei casi d'uso}

\textbf{L'analisi dei requisiti} può comprendere \textbf{storie o scenari} relativi al modo in cui l'applicazione può essere utilizzata dagli utenti;
queste storie possono essere scritte come \textbf{casi d'uso}. I casi d'uso \textbf{non sono un elaborato ad oggetti} ma semplicemente delle storie scritte. Sono tuttavia
uno strumento \textbf{diffuso nell'analisi dei requisiti}. Facciamo un'esempio: \newline
\textbf{Gioca una partita a Dadi}: Il giocatore chiede di lanciare i dadi. Il Sistema presenta il risultato: se il valore totale delle facce dei dadi è sette, il giocatore ha vinto; altrimenti ha perso.
\subsubsection{Definizione di un modello di dominio}
L'analisi orientata agli oggetti è interessata alla \textbf{creazione di una descrizione del dominio da un punto di vista ad oggetti}. Vengono identificati i \textbf{concetti, gli attributi e le associazioni considerati significativi}.
Il risultato può essere espresso come un \textbf{modello di dominio} che mostra i concetti o gli oggetti \textbf{significativi} del dominio. Esso è rappresentato nel seguente modo:
\begin{center}
    \includegraphics[width = 0.65\textwidth]{Images/7.png}
\end{center}
\subsubsection{Definizione dei diagrammi di interazione}
La \textbf{progettazione ad oggetti} è interessata alla \textbf{definizione di oggetti software, delle loro responsabilità e collaborazioni}. Una notazione comune per illustrare queste collaborazione è un
\textbf{diagramma di sequenza} (un tipo di diagramma UML). Esso mostra lo scambio di messaggi \textbf{tra oggetti software}, dunque l'invocazione di \textbf{metodi}. Esso è rappresentato nel seguente modo:
\begin{center}
    \includegraphics[width = 0.65\textwidth]{Images/8.png}
\end{center}
È interessante notare come \textbf{la progettazione degli oggetti software e dei programmi si può ispirare a un dominio del mondo reale}, tuttavia essa non è \textbf{nè un modello diretto nè una simulazione di questo dominio}. Quindi, per esempio,
seppur nel mondo reale è il giocatore a lanciare il dado, nel progetto software è l'oggetto \textit{PartitaADadi} che "lancia" i dadi.
\subsubsection{Definizione dei diagrammi di classe di progetto}

\end{document}
